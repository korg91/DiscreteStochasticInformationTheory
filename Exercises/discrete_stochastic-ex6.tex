\documentclass[12pt,a4paper]{report}

\usepackage{amsmath}
\usepackage{bbm}
\usepackage[utf8]{inputenc}
\usepackage{longtable}
\usepackage{amsthm}
\usepackage{amscd}
\usepackage{amssymb}
\usepackage{amsfonts}
\usepackage{amsmath}
\usepackage{mathtools}
\usepackage[shortlabels]{enumitem}
\usepackage[hyphens]{url}
\usepackage[scale=3]{ccicons}  % per le icone creative commons
\usepackage{hyperref}  % per i link nel pdf
\usepackage[rmargin=3.0cm,lmargin=3.0cm]{geometry}
%\usepackage{frontesp}  % prima pagina; il pacchetto frontesp.sty si trova nella stessa cartella del file .tex (deve essere adattato a mano)
\usepackage{setspace}  % per l'interlinea
\usepackage[english]{babel}  % per sillabazione
\usepackage[all]{xy} %diagrammi di funzioni
\usepackage{xspace} %per assicurare la corretta gestione degli spazi finali quando uso e.g. \AC. NB: sarebbe meglio trovare un'altra soluzione...cfr. http://tex.stackexchange.com/questions/15220/no-space-present-after-ensuremath
\usepackage{stmaryrd}
\usepackage{xfrac}
\usepackage{tikz-cd}
\usetikzlibrary{arrows,matrix,positioning,decorations.pathreplacing}
\usepackage{graphicx}
%\usepackage{parskip} %modifica la gestione degli spazi nei paragrafi, in particolare disabilita l'indentazione e aumenta lo spazio verticale tra i paragrafi



\theoremstyle{definition}
\newtheorem{theorem}{Theorem}[chapter] % resetta la numerazione dei teoremi per ogni capitolo
\newtheorem{corollary}[theorem]{Corollary} % la numerazione delle definizioni dipende da quella dei teoremi
\newtheorem{lemma}[theorem]{Lemma}
\newtheorem{proposition}[theorem]{Proposition}
\newtheorem{definition}[theorem]{Definition}
\newtheorem{Remark}[theorem]{Remark}
\newtheorem*{addendum}{Addendum}
\newtheorem*{examples}{Examples}
\newtheorem*{remark}{Remark}
\newtheorem*{remex}{Remarks and Examples}

%%% inizio comandi per stile per teoremi: "numero. Titolo" %%%
\newtheoremstyle{num.custom-title}
  {\topsep}   % ABOVESPACE
  {\topsep}   % BELOWSPACE
  {\normalfont}  % BODYFONT
  {0pt}       % INDENT (empty value is the same as 0pt)
  {\bfseries} % HEADFONT
  {}         % HEADPUNCT
  {5pt plus 1pt minus 1pt} % HEADSPACE
  {\thmnumber{#2.}\thmnote{ #3}}
  
\theoremstyle{num.custom-title}  
\newtheorem{teo_custom-title}[theorem]{} % per usarlo basta \begin{teo_custom-title}[<Titolo teorema>] (usa automaticamente la numerazione di [teo])
%%% fine comandi per stile per teoremi: "numero. Titolo" %%%

\newenvironment{claim}[1]{\par\noindent\underline{Claim#1:}\space}{} %per i claim
\newenvironment{claimproof}[1]{\par\noindent\underline{Proof:}\space#1}{\leavevmode\unskip\penalty9999 \hbox{}\nobreak\hfill\quad\hbox{$\blacksquare$}} %per le dimostrazioni dei claim

\DeclareMathOperator{\dom}{dom}
\DeclareMathOperator{\ran}{ran}
\DeclareMathOperator{\orb}{orb}
\DeclareMathOperator{\id}{id}
\DeclareMathOperator{\rk}{rk}
\DeclareMathOperator{\tor}{tor}
\let\o\relax % elimina \o dai comandi già definiti
\DeclareMathOperator{\o}{\mathsf{o}}
\let\Im\relax % elimina \o dai comandi già definiti
\DeclareMathOperator{\Im}{Im}
\DeclareMathOperator{\Zdv}{Zdv}
\DeclareMathOperator{\Hom}{Hom}
\DeclareMathOperator{\End}{End}
\DeclareMathOperator{\Ann}{Ann}
\DeclareMathOperator{\A}{\mathcal{A}}
\DeclareMathOperator{\B}{\mathcal{B}}
\DeclareMathOperator{\E}{\mathbb{E}}
\DeclareMathOperator{\PP}{\mathcal{P}}
\DeclareMathOperator{\LL}{\mathcal{L}}
\DeclareMathOperator{\Hrtg}{\text{Hrtg}}
\DeclareMathOperator{\Ord}{\text{Ord}}
\DeclareMathOperator{\J}{\mathcal{J}}
\DeclareMathOperator{\N}{\mathbb{N}}
\DeclareMathOperator{\R}{\mathbb{R}}
\DeclareMathOperator{\Z}{\mathbb{Z}}
\DeclareMathOperator{\U}{\mathfrak{U}}
\DeclareMathOperator{\PPP}{\mathbb{P}}
\DeclareMathOperator{\V}{\mathcal{V}}
\DeclareMathOperator{\Var}{Var}
\DeclareMathOperator{\Cov}{Cov}
\DeclareMathOperator{\a01}{\{0,1\}^{\star}}
\DeclareMathOperator{\imp}{\Rightarrow}
\DeclareMathOperator{\pmi}{\Leftarrow}
\DeclareMathOperator{\Pic}{Pic}
\DeclareMathOperator{\sm}{\setminus}
\DeclareMathOperator{\sse}{\subseteq}
\DeclareMathOperator{\cl}{cl}
\DeclareMathOperator{\Spec}{Spec}
\DeclareMathOperator{\Tr}{Tr}
\DeclareMathOperator{\spn}{span}
\DeclareMathOperator{\q}{\mathsf{q}}
\DeclareMathOperator{\h}{h}
\DeclareMathOperator{\GL}{GL}
\let\S\relax % elimina \S dai comandi già definiti
\DeclareMathOperator{\S}{S}
\DeclareMathOperator{\Cont}{Cont}
%\DeclareMathOperator{\gcd}{GCD}


\newcommand{\AC}{\ensuremath{\mathsf{AC}}\xspace}
\newcommand{\CC}{\ensuremath{\mathsf{CC}}\xspace}
\newcommand{\DC}{\ensuremath{\mathsf{DC}}\xspace}
\newcommand{\ZF}{\ensuremath{\mathsf{ZF}}\xspace}
\newcommand{\ZFC}{\ensuremath{\mathsf{ZFC}}\xspace}
\newcommand{\LS}{\ensuremath{\mathsf{LS}}\xspace}
\newcommand{\AMC}{\ensuremath{\mathsf{AMC}}\xspace}
\newcommand{\HRule}{\rule{\linewidth}{0.5mm}} %per la prima pagina
\newcommand{\qedblack}{\hfill $\blacksquare$}
\newcommand{\ol}{\overline}
\newcommand{\ul}{\underline}
\newcommand{\C}{\mathbb{C}}
\newcommand{\F}{\mathcal{F}}
\newcommand{\I}{\mathcal{I}}
\newcommand{\M}{\mathcal{M}}
\newcommand{\Q}{\mathbb{Q}}
\newcommand{\g}{\mathfrak{g}}
\newcommand{\p}{\mathfrak{p}}
\newcommand{\m}{\mathfrak{m}}
\newcommand{\T}{\mathcal{T}}
\newcommand{\X}{\mathbf{X}}
\newcommand{\x}{\mathbf{x}}
\newcommand{\IFF}{\Longleftrightarrow}
\newcommand{\RR}{\mathcal{R}}

\newcommand{\ndivides}{%
  \mathrel{\mkern.5mu % small adjustment
    % superimpose \nmid to \big|
    \ooalign{\hidewidth$\big|$\hidewidth\cr$\nmid$\cr}%
  }%
}

\renewcommand{\epsilon}{\varepsilon}
\renewcommand{\phi}{\varphi}
\renewcommand{\H}{\mathcal{H}}
%\renewcommand{\S}{\mathcal{S}}
\renewcommand{\1}{\mathbbm{1}}
\renewcommand{\O}{\mathcal{O}}
\renewcommand{\P}{\mathbb{P}}
\renewcommand{\u}{\mathbf{u}}
\renewcommand{\iff}{\Leftrightarrow}



%%%% INIZIO COMANDI PER EQUIVALENZE %%%%
\newcommand{\Implies}[2]{$\text{\ref{statement#1}}\!\implies\!\text{\ref{statement#2}}$}% X => Y
\newcommand{\punto}[1]{\item \label{statement#1}}


\newenvironment{equivalence}
    {\begin{enumerate}[label=(\arabic*),ref=(\arabic*)]
    }
    { 
	\end{enumerate}
    }
%%%% FINE COMANDI PER EQUIVALENZE %%%



% Interlinea 1.5
%\onehalfspacing  


%per le citazioni
\def\signed #1{{\leavevmode\unskip\nobreak\hfil\penalty50\hskip2em
  \hbox{}\nobreak\hfil(#1)%
  \parfillskip=0pt \finalhyphendemerits=0 \endgraf}}

\newsavebox\mybox
\newenvironment{aquote}[1]
  {\savebox\mybox{#1}\begin{quote}}
  {\signed{\usebox\mybox}\end{quote}}

%disabilita colore link
%\hypersetup{%
%    pdfborder = {0 0 0}
%}

\begin{document}

\noindent Andrea Gadotti \hfill 28/04/2015

\paragraph{Exercise 29.}

$(X_n)$ is not a Markov Chain. Indeed, consider $X_0$, $X_1$ and $X_2$. Suppose that $X_1=2$. This necessarily implies $Y_0=1$ and $Y_1=1$. Suppose \emph{also} $X_2=1$. This necessarily implies $Y_2=0$. It follows that under both these assumption, $X_3 \neq 2$, i.e.
\[
\P[X_3 = 2 | X_2=1, X_1=2] = 0.
\]
But
\begin{multline*}
\P[X_3 = 2 | X_2=1] = \frac{\P[X_3 = 2, X_2=1]}{\P[X_2=1]} = \\
\frac{\P[Y_3=1,Y_2=1,Y_1=0]}{\P[(Y_2=1 \wedge Y_1=0) \vee (Y_2=0 \wedge Y_1=1)]} = \frac{1/8}{1/2} = \frac{1}{4} \neq 0.
\end{multline*}


\paragraph{Exercise 30.}

We have 
\begin{itemize}
\item $\P[X_{n+1} = x-1 | X_n = x] = \left( \frac{x}{5} \right)^2$ for all $x=1,...,5$.
\item $\P[X_{n+1} = x | X_n = x] = \left( \frac{5-x}{5} \right) \cdot \left( \frac{x}{5} \right)$ for all $x=0,...,5$.
\item $\P[X_{n+1} = x+1 | X_n = x] = \left( \frac{5-x}{5} \right)^2$ for all $x=0,...,4$.
\item The transition probability is $0$ in any other case.
\end{itemize}


\paragraph{Exercise 31.}

\begin{enumerate}[(a)]
\item By the given transition probabilities we can find the missing ones:
\begin{itemize}
\item $p(4,0) = 2p(4,3) = 2/3 \cdot (1-1/4) = 1/2$.
\item $p(0,4) = 1-1/8-1/8 = 3/4$.
\item $p(2,0) = 1-1/4-1/4 = 1/2$.
\end{itemize}
Therefore the transition matrix $P = (p(i,j))_{i,j}$ is
\[
P =
\begin{pmatrix}
1/8 & 0 & 1/8 & 0 & 3/4 \\ 
0 & 1 & 0 & 0 & 0 \\ 
1/2 & 1/4 & 1/4 & 0 & 0 \\ 
0 & 0 & 0 & 1 & 0 \\ 
1/2 & 0 & 0 & 1/4 & 1/4 
\end{pmatrix}
\]
So, the transition graph is the following:

\begin{center}
\begin{tikzpicture}[->,>=stealth',shorten >=1pt,auto,node distance=3cm,
                    thick,main node/.style={circle,draw,font=\sffamily\Large\bfseries}]

  \node[main node] (0) at (90:4) {0};
  \node[main node] (1) at (90+72:4) {1};
  \node[main node] (2) at (90+2*72:4) {2};
  \node[main node] (3) at (90+3*72:4) {3};
  \node[main node] (4) at (90+4*72:4) {4};

  \path[every node/.style={font=\sffamily\small}]
     (	0) edge [loop above] node {1/8} (0)
     (	0) edge node [left] {1/8} (2)
     (	0) edge node [left] {3/4} (4)
     (1) edge [loop left] node {1} (1)
     (	2) edge [loop below] node {1/4} (2)
     (2) edge [bend left] node [left] {1/2} (0)
     (2) edge node [left] {1/4} (1)
     (	3) edge [loop below] node {1} (3)
     (4) edge [loop right] node {1/4} (4)
     (4) edge [bend right] node [right] {1/2} (0)
     (4) edge node [right] {1/4} (3);
\end{tikzpicture}
\end{center}

\item The Markov Chain is not irreducible because, for example, $p(1,1)=1$, i.e. $p(1,x)=0$ for any $x \neq 1$.
\item For every state $x$, $p(x,x)>0$. This means that there are no transient states, i.e. every state is recurrent.
\item The absorbing states are the ones such that $p(x,x)=1$. So $S = \{1,3\}$.
\item We trivially have $\P[X_n=3 \text{ for } n \geq 1 | X_0=2] \leq \P[X_1=3 | X_0=2] = p(2,3)=0$.
\item For all $n \geq 0$, define $Y_n := \1_{[X_n \not\in S | X_0=0]}$. Its expectation is
\begin{align*}
\E[Y_n] 
&= \P[X_n \not\in S | X_0=0] \\
&= \P[X_n=0 | X_0=0] + \P[X_n=2 | X_0=0] + \P[X_n=4 | X_0=0] \\
&= p^{(n)}(0,0) + p^{(n)}(0,2) + p^{(n)}(0,4).
\end{align*}
Now observe that
\[
T_0 = \sum_{n=0}^{\infty} Y_n
\]
almost surely, because $\P[X_{n+k} \not\in S, X_{n} \in S] = 0$ for all $k \geq 0$, since $S$ is the set of absorbing states.\\
Thus
\[
\E[T_0] = \sum_{n=0}^\infty \E[Y_n] = \sum_{n=0}^\infty \Big( p^{(n)}(0,0) + p^{(n)}(0,2) + p^{(n)}(0,4) \Big).
\]
\end{enumerate}






























\end{document}
