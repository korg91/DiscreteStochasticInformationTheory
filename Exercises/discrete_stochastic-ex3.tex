\documentclass[12pt,a4paper]{report}

\usepackage{amsmath}
\usepackage{bbm}
\usepackage[utf8]{inputenc}
\usepackage{longtable}
\usepackage{amsthm}
\usepackage{amscd}
\usepackage{amssymb}
\usepackage{amsfonts}
\usepackage{amsmath}
\usepackage{mathtools}
\usepackage[shortlabels]{enumitem}
\usepackage[hyphens]{url}
\usepackage[scale=3]{ccicons}  % per le icone creative commons
\usepackage{hyperref}  % per i link nel pdf
\usepackage[rmargin=3.0cm,lmargin=3.0cm]{geometry}
%\usepackage{frontesp}  % prima pagina; il pacchetto frontesp.sty si trova nella stessa cartella del file .tex (deve essere adattato a mano)
\usepackage{setspace}  % per l'interlinea
\usepackage[english]{babel}  % per sillabazione
\usepackage[all]{xy} %diagrammi di funzioni
\usepackage{xspace} %per assicurare la corretta gestione degli spazi finali quando uso e.g. \AC. NB: sarebbe meglio trovare un'altra soluzione...cfr. http://tex.stackexchange.com/questions/15220/no-space-present-after-ensuremath
\usepackage{stmaryrd}
\usepackage{xfrac}
\usepackage{tikz-cd}
\usetikzlibrary{matrix,positioning,decorations.pathreplacing}
\usepackage{graphicx}
%\usepackage{parskip} %modifica la gestione degli spazi nei paragrafi, in particolare disabilita l'indentazione e aumenta lo spazio verticale tra i paragrafi



\theoremstyle{definition}
\newtheorem{theorem}{Theorem}[chapter] % resetta la numerazione dei teoremi per ogni capitolo
\newtheorem{corollary}[theorem]{Corollary} % la numerazione delle definizioni dipende da quella dei teoremi
\newtheorem{lemma}[theorem]{Lemma}
\newtheorem{proposition}[theorem]{Proposition}
\newtheorem{definition}[theorem]{Definition}
\newtheorem{Remark}[theorem]{Remark}
\newtheorem*{addendum}{Addendum}
\newtheorem*{examples}{Examples}
\newtheorem*{remark}{Remark}
\newtheorem*{remex}{Remarks and Examples}

%%% inizio comandi per stile per teoremi: "numero. Titolo" %%%
\newtheoremstyle{num.custom-title}
  {\topsep}   % ABOVESPACE
  {\topsep}   % BELOWSPACE
  {\normalfont}  % BODYFONT
  {0pt}       % INDENT (empty value is the same as 0pt)
  {\bfseries} % HEADFONT
  {}         % HEADPUNCT
  {5pt plus 1pt minus 1pt} % HEADSPACE
  {\thmnumber{#2.}\thmnote{ #3}}
  
\theoremstyle{num.custom-title}  
\newtheorem{teo_custom-title}[theorem]{} % per usarlo basta \begin{teo_custom-title}[<Titolo teorema>] (usa automaticamente la numerazione di [teo])
%%% fine comandi per stile per teoremi: "numero. Titolo" %%%

\newenvironment{claim}[1]{\par\noindent\underline{Claim#1:}\space}{} %per i claim
\newenvironment{claimproof}[1]{\par\noindent\underline{Proof:}\space#1}{\leavevmode\unskip\penalty9999 \hbox{}\nobreak\hfill\quad\hbox{$\blacksquare$}} %per le dimostrazioni dei claim

\DeclareMathOperator{\dom}{dom}
\DeclareMathOperator{\ran}{ran}
\DeclareMathOperator{\orb}{orb}
\DeclareMathOperator{\id}{id}
\DeclareMathOperator{\rk}{rk}
\DeclareMathOperator{\tor}{tor}
\let\o\relax % elimina \o dai comandi già definiti
\DeclareMathOperator{\o}{\mathsf{o}}
\let\Im\relax % elimina \o dai comandi già definiti
\DeclareMathOperator{\Im}{Im}
\DeclareMathOperator{\Zdv}{Zdv}
\DeclareMathOperator{\Hom}{Hom}
\DeclareMathOperator{\End}{End}
\DeclareMathOperator{\Ann}{Ann}
\DeclareMathOperator{\A}{\mathcal{A}}
\DeclareMathOperator{\B}{\mathcal{B}}
\DeclareMathOperator{\E}{\mathbb{E}}
\DeclareMathOperator{\PP}{\mathcal{P}}
\DeclareMathOperator{\LL}{\mathcal{L}}
\DeclareMathOperator{\Hrtg}{\text{Hrtg}}
\DeclareMathOperator{\Ord}{\text{Ord}}
\DeclareMathOperator{\J}{\mathcal{J}}
\DeclareMathOperator{\N}{\mathbb{N}}
\DeclareMathOperator{\R}{\mathbb{R}}
\DeclareMathOperator{\Z}{\mathbb{Z}}
\DeclareMathOperator{\U}{\mathfrak{U}}
\DeclareMathOperator{\PPP}{\mathbb{P}}
\DeclareMathOperator{\V}{\mathcal{V}}
\DeclareMathOperator{\Var}{Var}
\DeclareMathOperator{\Cov}{Cov}
\DeclareMathOperator{\a01}{\{0,1\}^{\star}}
\DeclareMathOperator{\imp}{\Rightarrow}
\DeclareMathOperator{\pmi}{\Leftarrow}
\DeclareMathOperator{\Pic}{Pic}
\DeclareMathOperator{\sm}{\setminus}
\DeclareMathOperator{\sse}{\subseteq}
\DeclareMathOperator{\cl}{cl}
\DeclareMathOperator{\Spec}{Spec}
\DeclareMathOperator{\Tr}{Tr}
\DeclareMathOperator{\spn}{span}
\DeclareMathOperator{\q}{\mathsf{q}}
\DeclareMathOperator{\h}{h}
\DeclareMathOperator{\GL}{GL}
\let\S\relax % elimina \S dai comandi già definiti
\DeclareMathOperator{\S}{S}
\DeclareMathOperator{\Cont}{Cont}
%\DeclareMathOperator{\gcd}{GCD}


\newcommand{\AC}{\ensuremath{\mathsf{AC}}\xspace}
\newcommand{\CC}{\ensuremath{\mathsf{CC}}\xspace}
\newcommand{\DC}{\ensuremath{\mathsf{DC}}\xspace}
\newcommand{\ZF}{\ensuremath{\mathsf{ZF}}\xspace}
\newcommand{\ZFC}{\ensuremath{\mathsf{ZFC}}\xspace}
\newcommand{\LS}{\ensuremath{\mathsf{LS}}\xspace}
\newcommand{\AMC}{\ensuremath{\mathsf{AMC}}\xspace}
\newcommand{\HRule}{\rule{\linewidth}{0.5mm}} %per la prima pagina
\newcommand{\qedblack}{\hfill $\blacksquare$}
\newcommand{\ol}{\overline}
\newcommand{\ul}{\underline}
\newcommand{\C}{\mathbb{C}}
\newcommand{\F}{\mathcal{F}}
\newcommand{\I}{\mathcal{I}}
\newcommand{\M}{\mathcal{M}}
\newcommand{\Q}{\mathbb{Q}}
\newcommand{\g}{\mathfrak{g}}
\newcommand{\p}{\mathfrak{p}}
\newcommand{\m}{\mathfrak{m}}
\newcommand{\T}{\mathcal{T}}
\newcommand{\X}{\mathbf{X}}
\newcommand{\x}{\mathbf{x}}
\newcommand{\IFF}{\Longleftrightarrow}
\newcommand{\RR}{\mathcal{R}}

\newcommand{\ndivides}{%
  \mathrel{\mkern.5mu % small adjustment
    % superimpose \nmid to \big|
    \ooalign{\hidewidth$\big|$\hidewidth\cr$\nmid$\cr}%
  }%
}

\renewcommand{\epsilon}{\varepsilon}
\renewcommand{\phi}{\varphi}
\renewcommand{\H}{\mathcal{H}}
%\renewcommand{\S}{\mathcal{S}}
\renewcommand{\1}{\mathbbm{1}}
\renewcommand{\O}{\mathcal{O}}
\renewcommand{\P}{\mathbb{P}}
\renewcommand{\u}{\mathbf{u}}
\renewcommand{\iff}{\Leftrightarrow}



%%%% INIZIO COMANDI PER EQUIVALENZE %%%%
\newcommand{\Implies}[2]{$\text{\ref{statement#1}}\!\implies\!\text{\ref{statement#2}}$}% X => Y
\newcommand{\punto}[1]{\item \label{statement#1}}


\newenvironment{equivalence}
    {\begin{enumerate}[label=(\arabic*),ref=(\arabic*)]
    }
    { 
	\end{enumerate}
    }
%%%% FINE COMANDI PER EQUIVALENZE %%%



% Interlinea 1.5
%\onehalfspacing  


%per le citazioni
\def\signed #1{{\leavevmode\unskip\nobreak\hfil\penalty50\hskip2em
  \hbox{}\nobreak\hfil(#1)%
  \parfillskip=0pt \finalhyphendemerits=0 \endgraf}}

\newsavebox\mybox
\newenvironment{aquote}[1]
  {\savebox\mybox{#1}\begin{quote}}
  {\signed{\usebox\mybox}\end{quote}}

%disabilita colore link
%\hypersetup{%
%    pdfborder = {0 0 0}
%}

\begin{document}

\noindent Andrea Gadotti \hfill 21/04/2015

\paragraph{Exercise 13 (to be simplified).}
As for (a), we first prove that
\[
	\sum_{n=1}^\infty \P[X>n] = \sum_{k=1}^\infty k \P[k < X \leq k+1].
\]
Observe that 
\[
\{X>n\}= \biguplus_{k \geq n} \{k < X \leq k+1\}
\]
and thus 
\[
\P[X>n] = \sum_{k=n}^\infty \P[k < X \leq k+1].
\]
Therefore
\begin{multline*}
	\sum_{n=1}^\infty \P[X>n] = \sum_{n=1}^\infty \sum_{k=n}^\infty \P[k < X \leq k+1] = \sum_{k=1}^\infty \sum_{n=1}^k \P[k < X \leq k+1] =\\
	\sum_{k=1}^\infty \P[k < X \leq k+1] \sum_{n=1}^k 1 = \sum_{k=1}^\infty \P[k < X \leq k+1] \cdot k.
\end{multline*}
So the two sums are equal. Now we want to prove that 
\[
\E[X] < \infty \iff \sum_{k=1}^\infty k \P[k < X \leq k+1] < \infty.
\]
($\Longrightarrow$)
\begin{multline*}
	\infty > \int_\Omega X d\P > \int_\Omega \sum_{k=1}^\infty \left( k \cdot \1_{\{\omega : k<X(\omega) \leq k+1\}} \right) d\P= \sum_{k=1}^\infty \int_\Omega k \cdot \1_{\{\omega : k<X(\omega) \leq k+1\}} d\P=\\
	\sum_{k=1}^\infty k \int_\Omega \1_{\{\omega : k<X(\omega) \leq k+1\}} d\P = \sum_{k=1}^\infty k \P[k < X \leq k+1].
\end{multline*}
($\Longleftarrow$)
\begin{multline*}
	\int_\Omega X d\P \leq \int_\Omega \sum_{k=0}^\infty \left( (k+1) \cdot \1_{\{\omega : k<X(\omega) \leq k+1\}} \right) d\P = \sum_{k=0}^\infty \int_\Omega (k+1) \cdot \1_{\{\omega : k<X(\omega) \leq k+1\}} d\P=\\
	\sum_{k=0}^\infty (k+1) \int_\Omega \1_{\{\omega : k<X(\omega) \leq k+1\}} d\P = \sum_{k=0}^\infty (k+1) \P[k < X \leq k+1] < \infty.
\end{multline*}

And we are done.\\
As for (b), consider
\[
X_n := \underbrace{\sum_{x_k < n} x_k \1_{[X=x_k]}}_{=:Y_n} + n \1_{[X \geq n]}.
\]
Of course $X_n \nearrow X$, and so by monotone convergence
\[
\E[X] = \E[\lim_{n \to \infty} X_n] = \lim_{n \to \infty} \E[X_n] = \lim_{n \to \infty} \sum_{x_k < n} x_k \P[X=x_k] + n \P[X \geq n].
\]
Now observe that $\P[X=+\infty]>0$ implies $\E[X]=+\infty$ because $\P[X \geq n] \searrow \P[X=+\infty]$ by continuity. Since $\E[X<+\infty]$ by hypothesis, we trivially get $\P[X=+\infty]=0$. This means
\[
X \cdot \1_{[X<+\infty]} = X \text{ almost surely} \tag{$*$}
\]
Now, first observe that $Y_n \leq Y_{n+1}$ for all $n \in \N$. If $X(\omega) < \infty$, then $\exists n_0 \in \N$ s.t. $X(\omega) \leq n_0$. So $n \1_{[X \geq n]}(\omega)=0$ for all $n > n_0$, i.e. $X(\omega) = \lim_n Y_n(\omega)$. Thus $Y_n \nearrow X \cdot \1_{[X<+\infty]}$, i.e. $Y_n \nearrow X$ almost surely by $(*)$.\\
Hence by monotone convergence we obtain $\E[Y_n] \to \E[X]$ and $\E[X_n] \to \E[X]$. Since $\E[X]<+\infty$, we can finally conclude that $\E[X_n] - \E[Y_n] \to 0$, i.e. $n\P[X \geq n] \to 0$.

\paragraph{Exercise 14.} Observe that
\[
|Z-Y| = |X_n - Y - X_n + Z| \leq |X_n - Y| + |Z - X_n|
\]
by the triangular inequality. Thus
\begin{align*}
	\P[|Z-Y| \geq 2\epsilon]
	&\leq \P[|X_n - Y| + |Z - X_n| \geq 2\epsilon] \\
	&\leq \P[|X_n-Y| \geq \epsilon \text{ or } |X_n-Z| \geq \epsilon] \\
	&\leq \P[|X_n-Y| \geq \epsilon] + \P[|X_n-Z| \geq \epsilon] \to_{n \to \infty} 0+0=0
\end{align*}
for all $\epsilon>0$, i.e. $\P[|Z-Y| \geq 2\epsilon]=0$ for all $\epsilon>0$. Finally, observing that
\[
[Y \neq Z] = \bigcup_{k=1}^\infty \left[ |Y-Z| \geq \frac{1}{k} \right]
\]
we immediately conclude $\P[Y \neq Z]=0$ by subadditivity.

\paragraph{Exercise 15.} Observe that $\Var[S_n/n^p] = \frac{1}{n^{2p}} nC = \frac{1}{n^{2p-1}} C$. So, by Chebyshev's inequality, we get
\[
\P[|S_n/n^p| \geq a] \leq \frac{C}{n^{2p-1} a^2} \to 0
\]
because $2p-1>0$.

\paragraph{Exercise 16.} \ \\
\textbf{Solution 1.} This solution is way easier, and it follows immediately by the following Corollary we've seen in class (I didn't remember that at first):
\begin{center}
If $X_n$ are independent with finite variances and $\sum \frac{\Var[X]}{n^2} < \infty$, then $\sum \frac{X_n - \E[X_n]}{n}$ converges almost surely to a finite random variable.
\end{center}

\noindent\textbf{Solution 2.} Define $Y := \sum_{n=1}^\infty X_n/n$. First observe that $\E[X_n]=0$ and $\Var[X_n]=1$. Define $Y_m := \sum_{n=1}^m X_n/n$. We immediately have $\E[Y_m]=0$. Furthermore by independence
\[
\Var[Y_m] = \sum_{n=1}^m \frac{1}{n^2} \Var[X_n] = \sum_{n=1}^m \frac{1}{n^2}.
\]
%So, given the fact that $\lim_{m \to \infty} Y_m = \sum_{n=1}^\infty X_n/n$ exists (\textbf{which I seem not to be able to prove}), 
So by Chebyshev's inequality
\[
0 \leq \P[ |Y_m - \E[Y_m]| \geq k] \leq \frac{\Var[Y_m]}{k^2} = \frac{\sum_{n=1}^m \frac{1}{n^2}}{k^2}
\]
for all $a>0$. So
\[
0 \leq \lim_{m \to \infty} \P[ |Y_m| \geq k] \leq \lim_{m \to \infty} \frac{\sum_{n=1}^m \frac{1}{n^2}}{k^2} = \frac{1}{k^2} \sum_{n=1}^\infty \frac{1}{n^2} = \frac{1}{k^2} \frac{\pi}{6}.
\]
So $\lim_{m \to \infty} \P[ |Y_m| \geq k] \leq \frac{1}{k^2} \frac{\pi}{6}$, which implies (TO BE PROVEN) $\P[\lim_{m \to \infty} |Y_m| \geq k] \leq \frac{1}{k^2} \frac{\pi}{6}$. But $\lim_{m \to \infty} |Y_m| = |Y|$ (we are allowed to assume that the limit exists, see my question in the newsgroup), so $\P[|Y| \geq k] \leq \frac{1}{k^2} \frac{\pi}{6}$. Therefore
\[
\sum_{k=1}^\infty \P[|Y|>k] \leq \sum_{k=1}^\infty \frac{1}{k^2} \frac{\pi}{6} = \frac{\pi}{6} \sum_{k=1}^\infty \frac{1}{k^2} = \frac{\pi^2}{36} < \infty.
\]
By Exercise 16(a) this implies $\E[|Y|]<\infty$, which implies $\P[|Y|=+\infty]=0$ (see solution of Exercise 13). So $Y$ is finite a.s., i.e. $Y$ converges a.s.

\paragraph{Exercise 17.} Let's show that $\E[N_n/n] \to e^{-c}$ (the ``almost surely'' here makes no sense. $\E[N_n/n]$ is a number). It's trivial to check that $\E[N_n]=n(1-1/n)^r$. So $\E[N_n/n]=(1-1/n)^r$. Now, since $r/n \to c$,
\begin{multline*}
	\lim_{n \to +\infty} \left( 1-\frac{1}{n} \right)^r = \lim_{n \to +\infty} \left( 1-\frac{1}{n} \right)^{nc} = \lim_{n \to +\infty} \left( 1+\frac{1}{-n} \right)^{nc} = \lim_{n' \to -\infty} \left( 1+\frac{1}{n'} \right)^{-n' c} = \\
	\lim_{n' \to -\infty} \left( \left( 1+\frac{1}{n'} \right)^{n'} \right) ^{-c} = \left( \lim_{n' \to -\infty}  \left( 1+\frac{1}{n'} \right)^{n'} \right) ^{-c} = e^{-c}.
\end{multline*}
Now we are asked to show that $N_n/n \to e^{-c}$ in probability. We follow the hint. We will soon need the following:\\
\textbf{Lemma.} The following hold:
\begin{enumerate}
	\item $\Var \left[ \sum_{i=1}^n X_i \right] = \sum_{i=1}^n \Var[X_i] + 2\sum_{i<j} \Cov[X_i,X_j]$, where \\
	$\Cov[X,Y] := \E{\big[(X - \E[X])(Y - \E[Y])\big]}$.
	\item $\E[\1_A] = \P[A]$, $\Var[\1_A] = \P[A](1-\P[A])$, $\Cov[\1_A,\1_B]= \P[A \cap B] - \P[A]\P[B]$.
	\item $\displaystyle \sum_{\substack{i<j \\ j \leq n}} 1 = \sum_{j=1}^n \sum_{i=1}^{j-1} 1 = \frac{1}{2} (n^2-n)$.
\end{enumerate}
The proof of the lemma is very easy, therefore we skip it. Define $A_i := [i\text{-th box is empty}]$. Observe that $\P[A_i \cap A_j] = \P[A_j | A_i] \P[A_i] = \left( 1 - \frac{1}{n-1} \right)^r \P[A_i] \to e^{-2c}$. Observe also that $\P[A_i]=\P[A_j]$ for all $i,j \leq n$ and $\P[A_i | A_j] = \P[A_h | A_k]$ for all $i \neq j$, $h \neq k$. Putting everything together we get
\begin{align*}
	\Var[N_n]
	&= \Var \left[ \sum_{i=1}^n \1_{A_i} \right] \\
	&= \sum_{i=1}^n \Var[\1_{A_i}] + 2\sum_{i<j} \Cov[\1_{A_i},\1_{A_j}] \\
	&= \sum_{i=1}^n \P[A_i](1-\P[A_i]) + 2\sum_{i<j} ( \P[A_i \cap A_j] - \P[A_i]\P[A_j] ) \\
	&= \sum_{i=1}^n \P[A_i](1-\P[A_i]) + 2\sum_{i<j} ( \P[A_j | A_i] \P[A_i] - \P[A_i]\P[A_j] ) \\
	&= \P[A_i](1-\P[A_i]) \sum_{i=1}^n 1 + 2 (\P[A_j | A_i] \P[A_i] - \P[A_i]\P[A_j]) \sum_{i<j} 1 \\
	&= n \P[A_i](1-\P[A_i]) +  (n^2-n) (\P[A_j | A_i] \P[A_i] - \P[A_i]\P[A_j]).
\end{align*}
So
\begin{align*}
	\Var[N_n/n]
	&= \frac{1}{n^2} \Var[N_n] \\
	&= \frac{1}{n} \P[A_i] (1-\P[A_i] - \P[A_j | A_i] + \P[A_j]) +  (\P[A_j | A_i] \P[A_i] - \P[A_i]\P[A_j])\\
	&\to \underbrace{\frac{1}{n} e^{-c} (1- e^{-c} - e^{-c} + e^{-c})}_{\to 0} +  \underbrace{(e^{-2c} - e^{-2c})}_{=0}\\
	&\to 0.
\end{align*}

Finally, observe that by Chebyshev's inequality
\[
\P[ |N_n/n - \E[N_n/n]| \geq a] \leq \frac{\Var[N_n/n]}{a^2} \to 0
\]
for all $a>0$, i.e. $N_n/n \to \lim_{n \to \infty} \E[N_n/n] = e^{-c}$ in probability.












\end{document}
