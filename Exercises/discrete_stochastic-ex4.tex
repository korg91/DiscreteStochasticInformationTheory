\documentclass[12pt,a4paper]{report}

\usepackage{amsmath}
\usepackage{bbm}
\usepackage[utf8]{inputenc}
\usepackage{longtable}
\usepackage{amsthm}
\usepackage{amscd}
\usepackage{amssymb}
\usepackage{amsfonts}
\usepackage{amsmath}
\usepackage{mathtools}
\usepackage[shortlabels]{enumitem}
\usepackage[hyphens]{url}
\usepackage[scale=3]{ccicons}  % per le icone creative commons
\usepackage{hyperref}  % per i link nel pdf
\usepackage[rmargin=3.0cm,lmargin=3.0cm]{geometry}
%\usepackage{frontesp}  % prima pagina; il pacchetto frontesp.sty si trova nella stessa cartella del file .tex (deve essere adattato a mano)
\usepackage{setspace}  % per l'interlinea
\usepackage[english]{babel}  % per sillabazione
\usepackage[all]{xy} %diagrammi di funzioni
\usepackage{xspace} %per assicurare la corretta gestione degli spazi finali quando uso e.g. \AC. NB: sarebbe meglio trovare un'altra soluzione...cfr. http://tex.stackexchange.com/questions/15220/no-space-present-after-ensuremath
\usepackage{stmaryrd}
\usepackage{xfrac}
\usepackage{tikz-cd}
\usetikzlibrary{matrix,positioning,decorations.pathreplacing}
\usepackage{graphicx}
%\usepackage{parskip} %modifica la gestione degli spazi nei paragrafi, in particolare disabilita l'indentazione e aumenta lo spazio verticale tra i paragrafi



\theoremstyle{definition}
\newtheorem{theorem}{Theorem}[chapter] % resetta la numerazione dei teoremi per ogni capitolo
\newtheorem{corollary}[theorem]{Corollary} % la numerazione delle definizioni dipende da quella dei teoremi
\newtheorem{lemma}[theorem]{Lemma}
\newtheorem{proposition}[theorem]{Proposition}
\newtheorem{definition}[theorem]{Definition}
\newtheorem{Remark}[theorem]{Remark}
\newtheorem*{addendum}{Addendum}
\newtheorem*{examples}{Examples}
\newtheorem*{remark}{Remark}
\newtheorem*{remex}{Remarks and Examples}

%%% inizio comandi per stile per teoremi: "numero. Titolo" %%%
\newtheoremstyle{num.custom-title}
  {\topsep}   % ABOVESPACE
  {\topsep}   % BELOWSPACE
  {\normalfont}  % BODYFONT
  {0pt}       % INDENT (empty value is the same as 0pt)
  {\bfseries} % HEADFONT
  {}         % HEADPUNCT
  {5pt plus 1pt minus 1pt} % HEADSPACE
  {\thmnumber{#2.}\thmnote{ #3}}
  
\theoremstyle{num.custom-title}  
\newtheorem{teo_custom-title}[theorem]{} % per usarlo basta \begin{teo_custom-title}[<Titolo teorema>] (usa automaticamente la numerazione di [teo])
%%% fine comandi per stile per teoremi: "numero. Titolo" %%%

\newenvironment{claim}[1]{\par\noindent\underline{Claim#1:}\space}{} %per i claim
\newenvironment{claimproof}[1]{\par\noindent\underline{Proof:}\space#1}{\leavevmode\unskip\penalty9999 \hbox{}\nobreak\hfill\quad\hbox{$\blacksquare$}} %per le dimostrazioni dei claim

\DeclareMathOperator{\dom}{dom}
\DeclareMathOperator{\ran}{ran}
\DeclareMathOperator{\orb}{orb}
\DeclareMathOperator{\id}{id}
\DeclareMathOperator{\rk}{rk}
\DeclareMathOperator{\tor}{tor}
\let\o\relax % elimina \o dai comandi già definiti
\DeclareMathOperator{\o}{\mathsf{o}}
\let\Im\relax % elimina \o dai comandi già definiti
\DeclareMathOperator{\Im}{Im}
\DeclareMathOperator{\Zdv}{Zdv}
\DeclareMathOperator{\Hom}{Hom}
\DeclareMathOperator{\End}{End}
\DeclareMathOperator{\Ann}{Ann}
\DeclareMathOperator{\A}{\mathcal{A}}
\DeclareMathOperator{\B}{\mathcal{B}}
\DeclareMathOperator{\E}{\mathbb{E}}
\DeclareMathOperator{\PP}{\mathcal{P}}
\DeclareMathOperator{\LL}{\mathcal{L}}
\DeclareMathOperator{\Hrtg}{\text{Hrtg}}
\DeclareMathOperator{\Ord}{\text{Ord}}
\DeclareMathOperator{\J}{\mathcal{J}}
\DeclareMathOperator{\N}{\mathbb{N}}
\DeclareMathOperator{\R}{\mathbb{R}}
\DeclareMathOperator{\Z}{\mathbb{Z}}
\DeclareMathOperator{\U}{\mathfrak{U}}
\DeclareMathOperator{\PPP}{\mathbb{P}}
\DeclareMathOperator{\V}{\mathcal{V}}
\DeclareMathOperator{\Var}{Var}
\DeclareMathOperator{\Cov}{Cov}
\DeclareMathOperator{\a01}{\{0,1\}^{\star}}
\DeclareMathOperator{\imp}{\Rightarrow}
\DeclareMathOperator{\pmi}{\Leftarrow}
\DeclareMathOperator{\Pic}{Pic}
\DeclareMathOperator{\sm}{\setminus}
\DeclareMathOperator{\sse}{\subseteq}
\DeclareMathOperator{\cl}{cl}
\DeclareMathOperator{\Spec}{Spec}
\DeclareMathOperator{\Tr}{Tr}
\DeclareMathOperator{\spn}{span}
\DeclareMathOperator{\q}{\mathsf{q}}
\DeclareMathOperator{\h}{h}
\DeclareMathOperator{\GL}{GL}
\let\S\relax % elimina \S dai comandi già definiti
\DeclareMathOperator{\S}{S}
\DeclareMathOperator{\Cont}{Cont}
%\DeclareMathOperator{\gcd}{GCD}


\newcommand{\AC}{\ensuremath{\mathsf{AC}}\xspace}
\newcommand{\CC}{\ensuremath{\mathsf{CC}}\xspace}
\newcommand{\DC}{\ensuremath{\mathsf{DC}}\xspace}
\newcommand{\ZF}{\ensuremath{\mathsf{ZF}}\xspace}
\newcommand{\ZFC}{\ensuremath{\mathsf{ZFC}}\xspace}
\newcommand{\LS}{\ensuremath{\mathsf{LS}}\xspace}
\newcommand{\AMC}{\ensuremath{\mathsf{AMC}}\xspace}
\newcommand{\HRule}{\rule{\linewidth}{0.5mm}} %per la prima pagina
\newcommand{\qedblack}{\hfill $\blacksquare$}
\newcommand{\ol}{\overline}
\newcommand{\ul}{\underline}
\newcommand{\C}{\mathbb{C}}
\newcommand{\F}{\mathcal{F}}
\newcommand{\I}{\mathcal{I}}
\newcommand{\M}{\mathcal{M}}
\newcommand{\Q}{\mathbb{Q}}
\newcommand{\g}{\mathfrak{g}}
\newcommand{\p}{\mathfrak{p}}
\newcommand{\m}{\mathfrak{m}}
\newcommand{\T}{\mathcal{T}}
\newcommand{\X}{\mathbf{X}}
\newcommand{\x}{\mathbf{x}}
\newcommand{\IFF}{\Longleftrightarrow}
\newcommand{\RR}{\mathcal{R}}

\newcommand{\ndivides}{%
  \mathrel{\mkern.5mu % small adjustment
    % superimpose \nmid to \big|
    \ooalign{\hidewidth$\big|$\hidewidth\cr$\nmid$\cr}%
  }%
}

\renewcommand{\epsilon}{\varepsilon}
\renewcommand{\phi}{\varphi}
\renewcommand{\H}{\mathcal{H}}
%\renewcommand{\S}{\mathcal{S}}
\renewcommand{\1}{\mathbbm{1}}
\renewcommand{\O}{\mathcal{O}}
\renewcommand{\P}{\mathbb{P}}
\renewcommand{\u}{\mathbf{u}}
\renewcommand{\iff}{\Leftrightarrow}



%%%% INIZIO COMANDI PER EQUIVALENZE %%%%
\newcommand{\Implies}[2]{$\text{\ref{statement#1}}\!\implies\!\text{\ref{statement#2}}$}% X => Y
\newcommand{\punto}[1]{\item \label{statement#1}}


\newenvironment{equivalence}
    {\begin{enumerate}[label=(\arabic*),ref=(\arabic*)]
    }
    { 
	\end{enumerate}
    }
%%%% FINE COMANDI PER EQUIVALENZE %%%



% Interlinea 1.5
%\onehalfspacing  


%per le citazioni
\def\signed #1{{\leavevmode\unskip\nobreak\hfil\penalty50\hskip2em
  \hbox{}\nobreak\hfil(#1)%
  \parfillskip=0pt \finalhyphendemerits=0 \endgraf}}

\newsavebox\mybox
\newenvironment{aquote}[1]
  {\savebox\mybox{#1}\begin{quote}}
  {\signed{\usebox\mybox}\end{quote}}

%disabilita colore link
%\hypersetup{%
%    pdfborder = {0 0 0}
%}

\begin{document}

\noindent Andrea Gadotti \hfill 28/04/2015

\paragraph{Exercise 18.}
We must prove
\[
\phi \left( \sum_{i=1}^\infty x_i p(x_i) \right) \leq \sum_{i=1}^\infty \phi(x_i) p(x_i).
\]
First we deal with the finite form.
\begin{claim}{}
For all $N \in \N$,
\[
\phi \left( \frac{\sum_{i=1}^N x_i p(x_i)}{\sum_{i=1}^N p(x_i)} \right) \leq \frac{\sum_{i=1}^N \phi(x_i) p(x_i)}{\sum_{i=1}^N p(x_i)}. \tag{$*$}
\]
\begin{claimproof}
First recall that ``$\phi$ convex'' means
\[
\forall x_1, x_2 \in \R, \forall p_1,p_2 \in [0,1] \text{ s.t. } p_1+p_2=1 : \quad \phi(p_1 x_1 + p_2 x_2)\leq p_1 \phi(x_1) + p_2 \phi(x_2).
\]
We prove $(*)$ by induction on $N$. Define
\[
q(x_i) := \frac{p(x_i)}{\sum_{i=1}^N p(x_i)}.
\]
If we prove
\[
\phi \left( \sum_{i=1}^N x_i q(x_i) \right) \leq \sum_{i=1}^N \phi(x_i) q(x_i) \tag{$**$}
\]
we are done. Now observe that $q(x_i) \in [0,1]$ and $\sum_{i=1}^N q(x_i) = 1$. So:\\
If $N=2$, then $(**)$ is precisely the definition of ``$\phi$ convex''.\\
If $N>2$, this can be easily proven by ``grouping'' $q(x_1),...,q(x_N)$ and $q(x_N)$, and then applying the inductive hypothesis.
\end{claimproof}
\end{claim}
\\[6pt]
So $(*)$ holds for all $N \in \N$. Taking both sides to the limit we get
\[
\lim_{N \to \infty} \phi \left( \frac{\sum_{i=1}^N x_i p(x_i)}{\sum_{i=1}^N p(x_i)} \right) \leq \lim_{N \to \infty} \frac{\sum_{i=1}^N \phi(x_i) p(x_i)}{\sum_{i=1}^N p(x_i)} = \frac{\sum_{i=1}^\infty \phi(x_i) p(x_i)}{\sum_{i=1}^\infty p(x_i)} = \sum_{i=1}^\infty \phi(x_i) p(x_i).
\]
Recall that any convex function is continuous. So we can take the limit inside:
\[
\phi \left( \lim_{N \to \infty} \frac{\sum_{i=1}^N x_i p(x_i)}{\sum_{i=1}^N p(x_i)} \right) \leq \sum_{i=1}^\infty \phi(x_i) p(x_i),
\]
whereby
\[
\phi \left( \sum_{i=1}^\infty x_i p(x_i) \right) \leq \sum_{i=1}^\infty \phi(x_i) p(x_i),
\]
as wanted.


\paragraph{Exercise 19.}
\renewcommand{\X}{\mathcal{X}}
Let $\X \sse \R$ be the range of $X$. Consider $Y := f(X)$. For all $y \in f[\X]$ (\textbf{note:} $f^{-1}(y)$ is, a priori, \underline{a subset} of $\X$, not an element), define $p_Y(y) := p_X(f^{-1}(y))$. By definition of entropy we have
\[
H(f(X)) = -\sum_{y \in f[\X]} p_Y(y) \log_2 p_Y(y).
\]
Now observe that if $f$ is injective, then $\forall y \in f[\X] \ \exists! x \in \X : \ f^{-1}(y)=x$. So $p_Y(y)=p_X(f^{-1}(y))=p_X(x)$. This means
\[
H(f(X)) = -\sum_{y \in f[\mathcal{X}]} p_Y(y) \log_2 p_Y(y) = -\sum_{x \in \X} p_X(x) \log_2 p_X(x) = H(X).
\]
So, if $f$ is injective, then $H(f(X))=H(X)$.\\
Now suppose that $f$ is not injective, i.e. there exist at least two distinct elements $x_1,x_2 \in \X$ such that $f(x_1)=f(x_2) =: y$. Observe that $p_Y(y) = p_X(x_1)+p_X(x_2)$. Thus
\begin{multline*}
- p_Y(y) \log_2 p_Y(y) = - (p_X(x_1)+p_X(x_2)) \log_2 (p_X(x_1)+p_X(x_2))\\
= - p_X(x_1) \log_2 (p_X(x_1)+p_X(x_2)) - p_X(x_2) \log_2 (p_X(x_1)+p_X(x_2)) =: h(y),
\end{multline*}
while
\[
-\sum_{x \in \{x_1,x_2\}} p_X(x) \log_2 p_X(x) = - p_X(x_1) \log_2 p_X(x_1) - p_X(x_2) \log_2 (p_X(x_2)) =: h(x_1,x_2).
\]
Since $p_X(x_1)$ and $p_X(x_2)$ are positive and $\log_2$ is a monotone increasing function, we have that $\log_2 (p_X(x_1)+p_X(x_2)) \geq \log_2 p_X(x_1)$ and $\log_2 (p_X(x_1)+p_X(x_2)) \geq \log_2 p_X(x_2)$. Therefore $h(y) \leq h(x_1,x_2)$.\\
Since the addends of $H(f(X))$ are made like $h(y)$, and the ones of $H(X)$ are made like $h(x_1,x_2)$, repeating the same argument for all $y \in f(\X)$ we get that $H(f(X)) \leq H(X)$, as wanted.\\
\\
Given any random variable $Z$, it's immediate to see that every addend in $H(Z)$ is $\geq 0$. Hence, if $H(Z)=0$ then necessarily every addend is $0$. So $p_Z(z) \log_2 p_Z(z) = 0$ for all $z$. Since we can suppose w.l.o.g. $p_Z(z) \neq 0$, we get $\log_2 p_Z(z) = 1$, i.e. $p_Z(z)=1$.\\
So, $H(f(X))=0$ implies $p_Y(y)=1$ for some $y \in f[\mathcal{X}]$, i.e. $p_X(f^{-1}(y))=1$, i.e. $f[x]=y$ for almost every $x \in \X$. That is, $f$ is almost surely constant.


\paragraph{Exercise 20.} Define $Z := -\log_2 \sqrt[n]{\prod_{i=1}^n \P[X_i]}$. For all $i \in \N$, define $Y_i(\omega) := \log_2 \P[X_i=X_i(\omega)]$. Observe that
\begin{align*}
Z(\omega) 
&= -\log_2 \sqrt[n]{\prod_{i=1}^n \P[X_i(\omega)]} \\
&= -\frac{1}{n} \sum_{i=1}^n \log_2 \P[X_i(\omega)] \\
&= -\frac{1}{n} \sum_{i=1}^n \log_2 \P[X_i = X_i(\omega)] \\
&= -\frac{1}{n} \sum_{i=1}^n Y_i(\omega).
\end{align*}
Since the $X_i$'s are iid, it's immediate to check that also the $Y_i$'s are iid. Furthermore, write the range of $X_1$ as $\X_1 = \{x_1,x_2,...,x_k,...\}$. We trivially have
\[
\E[Y_1] = \sum_{k=1}^\infty \P[X_1=x_k] \log_2 \P[X_1=x_k] = -H(X_1).
\]
Therefore the statement follows directly by the Strong Law of Large Numbers.


\paragraph{Exercise 21.}

The (a) part has already been proven in the last paragraph of Exercise 19.\\
For the (b) part, for all $i=1,...,n$ define $p_i := \P[X=x_i]$. Now observe that by Exercise 22, taking $q_i=\frac{1}{n}$ for all $i=1,...,n$, we get
\[
H(X) = -\sum_{i=1}^n p_i \log_2 p_i \leq -\sum_{i=1}^n p_i \log_2 q_i =  -\sum_{i=1}^n p_i \log_2 \frac{1}{n} = \log_2 n \sum_{i=1}^n p_i = \log_2 n,
\]
and the equality holds iff $p_i=q_i=\frac{1}{n}$ for all $i=1,...,n$. So we are done.



\paragraph{Exercise 22.}

Since
\[
\log_2 a = \frac{ \ln a }{ \ln 2 }
\]
it is sufficient to prove the statement using the natural logarithm. Note that the natural logarithm satisfies $\ln x \leq x-1$ for all $x > 0$ \emph{with equality if and only if $x=1$}.\\
Of course we can suppose w.l.o.g. $p_i>0$ for all $i$.
So
\[
- \sum_{i=1}^n p_i \ln \frac{q_i}{p_i}  \geq  - \sum_{i=1}^n p_i \left( \frac{q_i}{p_i} - 1 \right) 
=  - \sum_{i=1}^n q_i + \sum_{i=1}^n p_i  
 =  - \sum_{i=1}^n q_i + 1 \geq  0. \tag{$*$}
\]
Therefore
\[
- \sum_{i=1}^n p_i \ln q_i \geq - \sum_{i=1}^n p_i \ln p_i,
\]
as wanted.\\
As for the equality, of course it holds if $p_i = q_i$ for all $i$. On the other hand, suppose
\[
- \sum_{i=1}^n p_i \ln q_i = - \sum_{i=1}^n p_i \ln p_i.
\]
Then $- \sum_{i=1}^n p_i \ln \frac{q_i}{p_i} =0$, and thus all the inequalities in $(*)$ are equalities, i.e.
\[
0 = \sum_{i=1}^n p_i \left( \frac{q_i}{p_i} - 1 \right) - \sum_{i=1}^n p_i \ln \frac{q_i}{p_i}
= \sum_{i=1}^n p_i \underbrace{\left( \frac{q_i}{p_i} - 1 - \ln \frac{q_i}{p_i} \right)}_{\geq 0 \text{ since } \ln x \leq x-1}.
\]
So we have a sum of positive numbers which equals $0$. Thus each term is zero, and since $p_i \neq 0$ for all $i$ we get
\[
\frac{q_i}{p_i} - 1 = \ln \frac{q_i}{p_i}
\]
for all $i$. So $\frac{q_i}{p_i}=1$ for all $i$, by the sentence written in italic at the top.


\paragraph{Exercise 23.}\ 
\begin{itemize}
\item[a)] It's clear that $\P[X=k]=2^{-k}$. First observe that for any $x \in [0,1)$ we have
\[
\sum_{k=1}^\infty k x^{k-1} = \sum_{k=0}^\infty k x^{k-1} = \left( \sum_{k=0}^\infty x^k \right)' = \left( \frac{1}{1-x} \right)' = \frac{1}{(1-x)^2}.
\]
So
\[
H(X) = -\sum_{k=1}^\infty 2^{-k} \log_2 2^{-k} = \sum_{k=1}^\infty k \left( \frac{1}{2} \right)^k = \frac{1}{2} \sum_{k=1}^\infty k \left( \frac{1}{2} \right)^{k-1} = \frac{1}{2} \frac{1}{1/4} = 2.
\]
\item[b)] Assume that the value of $X$ is written as the sequence (of coin tosses) it comes from. So it can be
\[
1=1, \quad 2=01, \quad 3=001, \quad 4=0001, \quad 5=00001, \quad \ldots
\]
Our sequence of questions will be:
\begin{enumerate}
\item Is the first bit $1$?
\item Is the second bit $1$?
\item Is the third bit $1$?
\item[$\vdots$]
\item[k.] Is the $k$-th bit $1$?
\item[$\vdots$]
\end{enumerate}
\end{itemize}
Observe that, at the first question whose answer is YES, we can stop, because we know that the value of $X$ is exactly the number of questions we asked so far.\\
Let's compute now the \emph{average} amount $\mu$ of required questions. The probability of $X=1$ is $1/2$, so with probability $1/2$ we will need just the first question. The probability of $X=2$ is $1/4$, so with probability $1/4$ we will need just the first two questions. In general
\[
\P[\text{the first $k$ question need to be asked}] = \frac{1}{2^k}.
\]
So
\[
\mu = \sum_{k=1}^\infty k \left( \frac{1}{2} \right)^k = 2,
\]
which is exactly $H(X)$ we found in point (a).































\end{document}
