\documentclass[12pt,a4paper]{report}

\usepackage{amsmath}
\usepackage{bbm}
\usepackage[utf8]{inputenc}
\usepackage{longtable}
\usepackage{amsthm}
\usepackage{amscd}
\usepackage{amssymb}
\usepackage{amsfonts}
\usepackage{amsmath}
\usepackage{mathtools}
\usepackage[shortlabels]{enumitem}
\usepackage[hyphens]{url}
\usepackage[scale=3]{ccicons}  % per le icone creative commons
\usepackage{hyperref}  % per i link nel pdf
\usepackage[rmargin=3.0cm,lmargin=3.0cm]{geometry}
%\usepackage{frontesp}  % prima pagina; il pacchetto frontesp.sty si trova nella stessa cartella del file .tex (deve essere adattato a mano)
\usepackage{setspace}  % per l'interlinea
\usepackage[english]{babel}  % per sillabazione
\usepackage[all]{xy} %diagrammi di funzioni
\usepackage{xspace} %per assicurare la corretta gestione degli spazi finali quando uso e.g. \AC. NB: sarebbe meglio trovare un'altra soluzione...cfr. http://tex.stackexchange.com/questions/15220/no-space-present-after-ensuremath
\usepackage{stmaryrd}
\usepackage{xfrac}
\usepackage{tikz-cd}
\usetikzlibrary{matrix,positioning,decorations.pathreplacing}
\usepackage{graphicx}
%\usepackage{parskip} %modifica la gestione degli spazi nei paragrafi, in particolare disabilita l'indentazione e aumenta lo spazio verticale tra i paragrafi



\theoremstyle{definition}
\newtheorem{theorem}{Theorem}[chapter] % resetta la numerazione dei teoremi per ogni capitolo
\newtheorem{corollary}[theorem]{Corollary} % la numerazione delle definizioni dipende da quella dei teoremi
\newtheorem{lemma}[theorem]{Lemma}
\newtheorem{proposition}[theorem]{Proposition}
\newtheorem{definition}[theorem]{Definition}
\newtheorem{Remark}[theorem]{Remark}
\newtheorem*{addendum}{Addendum}
\newtheorem*{examples}{Examples}
\newtheorem*{remark}{Remark}
\newtheorem*{remex}{Remarks and Examples}

%%% inizio comandi per stile per teoremi: "numero. Titolo" %%%
\newtheoremstyle{num.custom-title}
  {\topsep}   % ABOVESPACE
  {\topsep}   % BELOWSPACE
  {\normalfont}  % BODYFONT
  {0pt}       % INDENT (empty value is the same as 0pt)
  {\bfseries} % HEADFONT
  {}         % HEADPUNCT
  {5pt plus 1pt minus 1pt} % HEADSPACE
  {\thmnumber{#2.}\thmnote{ #3}}
  
\theoremstyle{num.custom-title}  
\newtheorem{teo_custom-title}[theorem]{} % per usarlo basta \begin{teo_custom-title}[<Titolo teorema>] (usa automaticamente la numerazione di [teo])
%%% fine comandi per stile per teoremi: "numero. Titolo" %%%

\newenvironment{claim}[1]{\par\noindent\underline{Claim#1:}\space}{} %per i claim
\newenvironment{claimproof}[1]{\par\noindent\underline{Proof:}\space#1}{\leavevmode\unskip\penalty9999 \hbox{}\nobreak\hfill\quad\hbox{$\blacksquare$}} %per le dimostrazioni dei claim

\DeclareMathOperator{\dom}{dom}
\DeclareMathOperator{\ran}{ran}
\DeclareMathOperator{\orb}{orb}
\DeclareMathOperator{\id}{id}
\DeclareMathOperator{\rk}{rk}
\DeclareMathOperator{\tor}{tor}
\let\o\relax % elimina \o dai comandi già definiti
\DeclareMathOperator{\o}{\mathsf{o}}
\let\Im\relax % elimina \o dai comandi già definiti
\DeclareMathOperator{\Im}{Im}
\DeclareMathOperator{\Zdv}{Zdv}
\DeclareMathOperator{\Hom}{Hom}
\DeclareMathOperator{\End}{End}
\DeclareMathOperator{\Ann}{Ann}
\DeclareMathOperator{\A}{\mathcal{A}}
\DeclareMathOperator{\B}{\mathcal{B}}
\DeclareMathOperator{\E}{\mathbb{E}}
\DeclareMathOperator{\PP}{\mathcal{P}}
\DeclareMathOperator{\LL}{\mathcal{L}}
\DeclareMathOperator{\Hrtg}{\text{Hrtg}}
\DeclareMathOperator{\Ord}{\text{Ord}}
\DeclareMathOperator{\J}{\mathcal{J}}
\DeclareMathOperator{\N}{\mathbb{N}}
\DeclareMathOperator{\R}{\mathbb{R}}
\DeclareMathOperator{\Z}{\mathbb{Z}}
\DeclareMathOperator{\U}{\mathfrak{U}}
\DeclareMathOperator{\PPP}{\mathbb{P}}
\DeclareMathOperator{\V}{\mathcal{V}}
\DeclareMathOperator{\Var}{Var}
\DeclareMathOperator{\Cov}{Cov}
\DeclareMathOperator{\a01}{\{0,1\}^{\star}}
\DeclareMathOperator{\imp}{\Rightarrow}
\DeclareMathOperator{\pmi}{\Leftarrow}
\DeclareMathOperator{\Pic}{Pic}
\DeclareMathOperator{\sm}{\setminus}
\DeclareMathOperator{\sse}{\subseteq}
\DeclareMathOperator{\cl}{cl}
\DeclareMathOperator{\Spec}{Spec}
\DeclareMathOperator{\Tr}{Tr}
\DeclareMathOperator{\spn}{span}
\DeclareMathOperator{\q}{\mathsf{q}}
\DeclareMathOperator{\h}{h}
\DeclareMathOperator{\GL}{GL}
\let\S\relax % elimina \S dai comandi già definiti
\DeclareMathOperator{\S}{S}
\DeclareMathOperator{\Cont}{Cont}
%\DeclareMathOperator{\gcd}{GCD}


\newcommand{\AC}{\ensuremath{\mathsf{AC}}\xspace}
\newcommand{\CC}{\ensuremath{\mathsf{CC}}\xspace}
\newcommand{\DC}{\ensuremath{\mathsf{DC}}\xspace}
\newcommand{\ZF}{\ensuremath{\mathsf{ZF}}\xspace}
\newcommand{\ZFC}{\ensuremath{\mathsf{ZFC}}\xspace}
\newcommand{\LS}{\ensuremath{\mathsf{LS}}\xspace}
\newcommand{\AMC}{\ensuremath{\mathsf{AMC}}\xspace}
\newcommand{\HRule}{\rule{\linewidth}{0.5mm}} %per la prima pagina
\newcommand{\qedblack}{\hfill $\blacksquare$}
\newcommand{\ol}{\overline}
\newcommand{\ul}{\underline}
\newcommand{\C}{\mathbb{C}}
\newcommand{\F}{\mathcal{F}}
\newcommand{\I}{\mathcal{I}}
\newcommand{\M}{\mathcal{M}}
\newcommand{\Q}{\mathbb{Q}}
\newcommand{\g}{\mathfrak{g}}
\newcommand{\p}{\mathfrak{p}}
\newcommand{\m}{\mathfrak{m}}
\newcommand{\T}{\mathcal{T}}
\newcommand{\X}{\mathbf{X}}
\newcommand{\x}{\mathbf{x}}
\newcommand{\IFF}{\Longleftrightarrow}
\newcommand{\RR}{\mathcal{R}}

\newcommand{\ndivides}{%
  \mathrel{\mkern.5mu % small adjustment
    % superimpose \nmid to \big|
    \ooalign{\hidewidth$\big|$\hidewidth\cr$\nmid$\cr}%
  }%
}

\renewcommand{\epsilon}{\varepsilon}
\renewcommand{\phi}{\varphi}
\renewcommand{\H}{\mathcal{H}}
%\renewcommand{\S}{\mathcal{S}}
\renewcommand{\1}{\mathbbm{1}}
\renewcommand{\O}{\mathcal{O}}
\renewcommand{\P}{\mathbb{P}}
\renewcommand{\u}{\mathbf{u}}
\renewcommand{\iff}{\Leftrightarrow}



%%%% INIZIO COMANDI PER EQUIVALENZE %%%%
\newcommand{\Implies}[2]{$\text{\ref{statement#1}}\!\implies\!\text{\ref{statement#2}}$}% X => Y
\newcommand{\punto}[1]{\item \label{statement#1}}


\newenvironment{equivalence}
    {\begin{enumerate}[label=(\arabic*),ref=(\arabic*)]
    }
    { 
	\end{enumerate}
    }
%%%% FINE COMANDI PER EQUIVALENZE %%%



% Interlinea 1.5
%\onehalfspacing  


%per le citazioni
\def\signed #1{{\leavevmode\unskip\nobreak\hfil\penalty50\hskip2em
  \hbox{}\nobreak\hfil(#1)%
  \parfillskip=0pt \finalhyphendemerits=0 \endgraf}}

\newsavebox\mybox
\newenvironment{aquote}[1]
  {\savebox\mybox{#1}\begin{quote}}
  {\signed{\usebox\mybox}\end{quote}}

%disabilita colore link
%\hypersetup{%
%    pdfborder = {0 0 0}
%}

\begin{document}

\noindent Andrea Gadotti \hfill 17/03/2015

\paragraph{Exercise 1.}

\begin{enumerate}[(a)]
\item $\A$ must contain $\emptyset$, $\Omega$, $\{1,3,5\}$ and $\{2,4,6\}$. Such a collection is already closed under union, intersection and complement. So it's precisely $\A$.
\item $\A$ must contain $\emptyset$, $\Omega$, $\{1,2\}$, $\{3,4\}$ and $\{5,6\}$, thus (using the union) also $\{1,2,3,4\}$, $\{1,2,5,6\}$ and $\{3,4,5,6\}$. It's immediate to check that such a collection is already closed under intersection and complement. So it's precisely $\A$.
\item $\A$ must contain every singleton, and thus $\A = \PP(\Omega)$, because $\Omega$ is finite and thus every subset can be written as the \emph{countable} (finite) union of the singletons of its elements.
\end{enumerate}

\paragraph{Exercise 2.} Let $\Omega := \{r,b,g,w\}$. We can construct the $\sigma$-algebra formed by $\emptyset,\Omega$ and:
\begin{enumerate}
\item nothing else.
\item One singleton, and thus also its complement. These are 4 different algebras (one for every singleton).
\item One set of cardinality 2, and thus also its complement. These are $6/2=3$ different algebras (there are 6 choices for sets of cardinality 2, but every such a set determines also the other one).
\item \emph{One set of cardinality 3: these algebras are exactly the same of the second point.}
\item \emph{One set of cardinality 4: same of first point.}
\item Two singletons, and thus also their complements, their union and the complement of their union. These are 6 different algebras (one for every choice of two singletons).
\item Two sets of cardinality 2. \emph{If the two sets are disjoint, the algebras are the same of third point.} If the two sets are not disjoint, then their intersection is a singleton. Now it's easy to check that we can construct every singleton, and thus these algebras are all $\PP(\Omega)$. Thus we have found just 1 more algebra.
\item \emph{Two sets of cardinality 3: same of sixth point.}
\item \emph{Three singletons: it's immediate to construct the other singleton, and thus we obtain $\PP(\Omega)$.}
\item \emph{Three sets of cardinality 2: it's easy to check that in this case we can always construct every singleton, and thus we get $\PP(\Omega)$.}
\item \emph{Three sets of cardinality 3: it's easy to check that in this case we can always construct every singleton, and thus we get $\PP(\Omega)$.}
\item \emph{Four singletons: this is trivially $\PP(\Omega)$.}
\end{enumerate}

So the different $\sigma$-algebras are 15. The non-isomorphic ones are as many as the non-italic points, i.e. 5.

\paragraph{Exercise 3.} 
\begin{enumerate}[(a)]
\item $\P[A \cap B^c] = \P[A] - \P[A \cap B] = \P[A] - \P[A]\P[B] = \P[A](1-\P[B])=\P[A]\P[B^c]$.
\item $\P[A^c \cap B^c] = \P[B^c] - \P[A \cap B^c] = \P[B^c] - \P[A]\P[B^c] = \P[B^c](1-\P[A]) = \P[B^c]\P[A^c]$.
\end{enumerate}

\paragraph{Exercise 4.}\ \\
\textbf{Note:} the following solution might look too complex/heavy/elaborated. I am pretty sure there aren't much easier ways to solve the exercise, but I could be wrong. Nevertheless, one thing is for sure: \textbf{answers of the type ``if the firstborn is a girl, then the chance that the second is a boy is 1/2, because the events are \emph{clearly} independent'' MAKE. NO. SENSE.} The exercise is given \emph{precisely} to force us to set up a theoretical environment which models the real problem, but where we can \emph{formally prove} the statements. And indeed such independence can be \emph{proved} in the following model. Of course, one can argue that my probability space doesn't faithfully model the real problem, but that's a different story.\\
\\
\noindent\textbf{Solution.} We define $\Omega := \{0,1\} \times \{0,1\} \times \{0,1\}$. By $0$ we mean boy and by $1$ we mean ``girl''.
\begin{enumerate}[(a)]
\item We take $\A := \PP(\Omega)$ and $\P[\omega] := \frac{1}{2^3}$ for all $\omega \in \Omega$. Since $\Omega$ is finite, this completely determines $\P$ on $\A$ (by the additive property).
\item Recall that $\P[A|B] = \P[A \cap B]/\P[B]$. Let $A := \{(x,y,z) \in \Omega \mid y=0\}$ and $B := \{(x,y,z) \in \Omega \mid x=1\}$. Then the exercise asks us to to compute $\P[A|B]$, and it's very easy to check that $\P[A \cap B] = \frac{1}{2^2}$ and $\P[B]=\frac{1}{2}$, thus the result is $\frac{1}{2}$.
\item Define $A := \{(x,y,z) \in \Omega \mid x=y=z=1\} = \{(1,1,1)\}$ and $B := \{(x,y,z) \in \Omega \mid x=1 \vee y=1 \vee z=1\}$. We shall compute $\P[A|B] = \P[A \cap B]/\P[B]$. Observe that $\P[A] = \frac{1}{2^3}$. Furthermore, trivially $A \sse B$, thus $\P[A \cap B] = \P[A] = \frac{1}{2^3}$. Now observe that $\P[B]=1-\P[B^c]$, and trivially $B^c = \{(0,0,0)\}$. Thus $\P[B]=1-\frac{1}{2^3}$. So
\[
\P[A|B] = \frac{1}{2^3} \cdot \frac{2^3}{2^3-1} = \frac{1}{7}.
\]
\end{enumerate}

\paragraph{Exercise 5.}
\begin{enumerate}[(a)]
\item Let $A$ denote the event ``$X=4$'' and $B$ denote the event ``all coins show head''. We are asked to find $\P[A|B]$. By Bayes theorem we know that
\[
\P[A|B] = \frac{\P[B|A] \P[B]}{\P[A]}.
\]
Now observe that (after setting up an appropriate abstract model for this problem we could prove that) $\P[A] = \frac{16}{110}$ and $\P[B|A] = \frac{1}{2^4}$. It is left to compute $\P[B]$. Let $C_k$ be the event ``$X=k$''. Of course such events are pairwise disjoint and they cover $\Omega$. So by the rule of total probability we have
\[
\P[B] = \sum_{k=1}^{10} \P[B|C_k] \P[C_k] = \sum_{k=1}^{10} \frac{1}{2^k} \frac{2k}{110} = \frac{1}{110} \sum_{k=1}^{10} k \left( \frac{1}{2} \right)^{k-1} = \frac{1}{110} \frac{509}{128}.
\]
So
\[
\P[A|B] = \frac{1}{2^4} \frac{1}{110} \frac{509}{128} \frac{110}{16} = \frac{509}{2^{15}}.
\]
\item Let $D$ be the event ``$X$ is even''. We are asked to check whether
\[
\P[D \cap B] = \P[D] \P[B].
\]
%Of course $\P[D] = (4 + 8 + 12)/110 = \frac{12}{55}$. 
We already know $\P[B]$. Let's compute $\P[D \cap B]$. Observe that
\begin{align*}
\P[D \cap B] 
&= \P[B|D] \P[D] \\
&= \P[B | C_2 \uplus C_4 \uplus C_6] \P[D] \\ 
&= \Big( \P[B | C_2] + \P[B | C_4] + \P[B | C_6] \Big) \P[D] \tag{immediate to check} \\
&= \Big( \frac{1}{2^2} + \frac{1}{2^4} + \frac{1}{2^6} \Big) \P[D] \\
&= \frac{25}{64} \P[D].
\end{align*}
But $\P[B] \neq \frac{25}{64}$, and thus $D$ and $B$ are not independent.
\end{enumerate}

























\end{document}
