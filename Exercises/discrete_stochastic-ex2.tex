\documentclass[12pt,a4paper]{report}

\usepackage{amsmath}
\usepackage{bbm}
\usepackage[utf8]{inputenc}
\usepackage{longtable}
\usepackage{amsthm}
\usepackage{amscd}
\usepackage{amssymb}
\usepackage{amsfonts}
\usepackage{amsmath}
\usepackage{mathtools}
\usepackage[shortlabels]{enumitem}
\usepackage[hyphens]{url}
\usepackage[scale=3]{ccicons}  % per le icone creative commons
\usepackage{hyperref}  % per i link nel pdf
\usepackage[rmargin=3.0cm,lmargin=3.0cm]{geometry}
%\usepackage{frontesp}  % prima pagina; il pacchetto frontesp.sty si trova nella stessa cartella del file .tex (deve essere adattato a mano)
\usepackage{setspace}  % per l'interlinea
\usepackage[english]{babel}  % per sillabazione
\usepackage[all]{xy} %diagrammi di funzioni
\usepackage{xspace} %per assicurare la corretta gestione degli spazi finali quando uso e.g. \AC. NB: sarebbe meglio trovare un'altra soluzione...cfr. http://tex.stackexchange.com/questions/15220/no-space-present-after-ensuremath
\usepackage{stmaryrd}
\usepackage{xfrac}
\usepackage{tikz-cd}
\usetikzlibrary{matrix,positioning,decorations.pathreplacing}
\usepackage{graphicx}
%\usepackage{parskip} %modifica la gestione degli spazi nei paragrafi, in particolare disabilita l'indentazione e aumenta lo spazio verticale tra i paragrafi



\theoremstyle{definition}
\newtheorem{theorem}{Theorem}[chapter] % resetta la numerazione dei teoremi per ogni capitolo
\newtheorem{corollary}[theorem]{Corollary} % la numerazione delle definizioni dipende da quella dei teoremi
\newtheorem{lemma}[theorem]{Lemma}
\newtheorem{proposition}[theorem]{Proposition}
\newtheorem{definition}[theorem]{Definition}
\newtheorem{Remark}[theorem]{Remark}
\newtheorem*{addendum}{Addendum}
\newtheorem*{examples}{Examples}
\newtheorem*{remark}{Remark}
\newtheorem*{remex}{Remarks and Examples}

%%% inizio comandi per stile per teoremi: "numero. Titolo" %%%
\newtheoremstyle{num.custom-title}
  {\topsep}   % ABOVESPACE
  {\topsep}   % BELOWSPACE
  {\normalfont}  % BODYFONT
  {0pt}       % INDENT (empty value is the same as 0pt)
  {\bfseries} % HEADFONT
  {}         % HEADPUNCT
  {5pt plus 1pt minus 1pt} % HEADSPACE
  {\thmnumber{#2.}\thmnote{ #3}}
  
\theoremstyle{num.custom-title}  
\newtheorem{teo_custom-title}[theorem]{} % per usarlo basta \begin{teo_custom-title}[<Titolo teorema>] (usa automaticamente la numerazione di [teo])
%%% fine comandi per stile per teoremi: "numero. Titolo" %%%

\newenvironment{claim}[1]{\par\noindent\underline{Claim#1:}\space}{} %per i claim
\newenvironment{claimproof}[1]{\par\noindent\underline{Proof:}\space#1}{\leavevmode\unskip\penalty9999 \hbox{}\nobreak\hfill\quad\hbox{$\blacksquare$}} %per le dimostrazioni dei claim

\DeclareMathOperator{\dom}{dom}
\DeclareMathOperator{\ran}{ran}
\DeclareMathOperator{\orb}{orb}
\DeclareMathOperator{\id}{id}
\DeclareMathOperator{\rk}{rk}
\DeclareMathOperator{\tor}{tor}
\let\o\relax % elimina \o dai comandi già definiti
\DeclareMathOperator{\o}{\mathsf{o}}
\let\Im\relax % elimina \o dai comandi già definiti
\DeclareMathOperator{\Im}{Im}
\DeclareMathOperator{\Zdv}{Zdv}
\DeclareMathOperator{\Hom}{Hom}
\DeclareMathOperator{\End}{End}
\DeclareMathOperator{\Ann}{Ann}
\DeclareMathOperator{\A}{\mathcal{A}}
\DeclareMathOperator{\B}{\mathcal{B}}
\DeclareMathOperator{\E}{\mathbb{E}}
\DeclareMathOperator{\PP}{\mathcal{P}}
\DeclareMathOperator{\LL}{\mathcal{L}}
\DeclareMathOperator{\Hrtg}{\text{Hrtg}}
\DeclareMathOperator{\Ord}{\text{Ord}}
\DeclareMathOperator{\J}{\mathcal{J}}
\DeclareMathOperator{\N}{\mathbb{N}}
\DeclareMathOperator{\R}{\mathbb{R}}
\DeclareMathOperator{\Z}{\mathbb{Z}}
\DeclareMathOperator{\U}{\mathfrak{U}}
\DeclareMathOperator{\PPP}{\mathbb{P}}
\DeclareMathOperator{\V}{\mathcal{V}}
\DeclareMathOperator{\Var}{Var}
\DeclareMathOperator{\Cov}{Cov}
\DeclareMathOperator{\a01}{\{0,1\}^{\star}}
\DeclareMathOperator{\imp}{\Rightarrow}
\DeclareMathOperator{\pmi}{\Leftarrow}
\DeclareMathOperator{\Pic}{Pic}
\DeclareMathOperator{\sm}{\setminus}
\DeclareMathOperator{\sse}{\subseteq}
\DeclareMathOperator{\cl}{cl}
\DeclareMathOperator{\Spec}{Spec}
\DeclareMathOperator{\Tr}{Tr}
\DeclareMathOperator{\spn}{span}
\DeclareMathOperator{\q}{\mathsf{q}}
\DeclareMathOperator{\h}{h}
\DeclareMathOperator{\GL}{GL}
\let\S\relax % elimina \S dai comandi già definiti
\DeclareMathOperator{\S}{S}
\DeclareMathOperator{\Cont}{Cont}
%\DeclareMathOperator{\gcd}{GCD}


\newcommand{\AC}{\ensuremath{\mathsf{AC}}\xspace}
\newcommand{\CC}{\ensuremath{\mathsf{CC}}\xspace}
\newcommand{\DC}{\ensuremath{\mathsf{DC}}\xspace}
\newcommand{\ZF}{\ensuremath{\mathsf{ZF}}\xspace}
\newcommand{\ZFC}{\ensuremath{\mathsf{ZFC}}\xspace}
\newcommand{\LS}{\ensuremath{\mathsf{LS}}\xspace}
\newcommand{\AMC}{\ensuremath{\mathsf{AMC}}\xspace}
\newcommand{\HRule}{\rule{\linewidth}{0.5mm}} %per la prima pagina
\newcommand{\qedblack}{\hfill $\blacksquare$}
\newcommand{\ol}{\overline}
\newcommand{\ul}{\underline}
\newcommand{\C}{\mathbb{C}}
\newcommand{\F}{\mathcal{F}}
\newcommand{\I}{\mathcal{I}}
\newcommand{\M}{\mathcal{M}}
\newcommand{\Q}{\mathbb{Q}}
\newcommand{\g}{\mathfrak{g}}
\newcommand{\p}{\mathfrak{p}}
\newcommand{\m}{\mathfrak{m}}
\newcommand{\T}{\mathcal{T}}
\newcommand{\X}{\mathbf{X}}
\newcommand{\x}{\mathbf{x}}
\newcommand{\IFF}{\Longleftrightarrow}
\newcommand{\RR}{\mathcal{R}}

\newcommand{\ndivides}{%
  \mathrel{\mkern.5mu % small adjustment
    % superimpose \nmid to \big|
    \ooalign{\hidewidth$\big|$\hidewidth\cr$\nmid$\cr}%
  }%
}

\renewcommand{\epsilon}{\varepsilon}
\renewcommand{\phi}{\varphi}
\renewcommand{\H}{\mathcal{H}}
%\renewcommand{\S}{\mathcal{S}}
\renewcommand{\1}{\mathbbm{1}}
\renewcommand{\O}{\mathcal{O}}
\renewcommand{\P}{\mathbb{P}}
\renewcommand{\u}{\mathbf{u}}
\renewcommand{\iff}{\Leftrightarrow}



%%%% INIZIO COMANDI PER EQUIVALENZE %%%%
\newcommand{\Implies}[2]{$\text{\ref{statement#1}}\!\implies\!\text{\ref{statement#2}}$}% X => Y
\newcommand{\punto}[1]{\item \label{statement#1}}


\newenvironment{equivalence}
    {\begin{enumerate}[label=(\arabic*),ref=(\arabic*)]
    }
    { 
	\end{enumerate}
    }
%%%% FINE COMANDI PER EQUIVALENZE %%%



% Interlinea 1.5
%\onehalfspacing  


%per le citazioni
\def\signed #1{{\leavevmode\unskip\nobreak\hfil\penalty50\hskip2em
  \hbox{}\nobreak\hfil(#1)%
  \parfillskip=0pt \finalhyphendemerits=0 \endgraf}}

\newsavebox\mybox
\newenvironment{aquote}[1]
  {\savebox\mybox{#1}\begin{quote}}
  {\signed{\usebox\mybox}\end{quote}}

%disabilita colore link
%\hypersetup{%
%    pdfborder = {0 0 0}
%}

\begin{document}

\noindent Andrea Gadotti \hfill 24/03/2015

\paragraph{Exercise 6.}
\begin{proof}
Suppose $\sum^{\infty}_{n=1}\P(A_n)=\infty$. We want to show $1-\P(\limsup_{n \rightarrow \infty} A_n) = 0$. First of all note that for all $N \in \N$ we have $\sum^{\infty}_{n=N}\P(A_n)=\infty$. Now observe that
\begin{multline*}
1 - \P(\limsup_{n \rightarrow \infty} A_n)  = 
1 - \P\left(\left(\bigcap_{N=1}^{\infty} \bigcup_{n=N}^{\infty}A_n\right)\right) =
\P\left(\left(\bigcap_{N=1}^{\infty} \bigcup_{n=N}^{\infty}A_n\right)^c \, \right) =\\
=\P\left(\bigcup_{N=1}^{\infty} \bigcap_{n=N}^{\infty}A_n^{c}\right)
= \P\left(\liminf_{n \rightarrow \infty}A_n^{c}\right)= \lim_{N \rightarrow \infty}\P\left(\bigcap_{n=N}^{\infty}A_n^{c}\right).
\end{multline*}
So it is enough to show that $\P\left(\bigcap_{n=N}^{\infty}A_n^{c}\right) = 0$ for all $N \in \N$. Since the $(A_n)^{\infty}_{n = 1}$ are independent and $1-x \leq e^{-x}$ for all $x \in \R^+$:
\begin{align*}
\P\left(\bigcap_{n=N}^{\infty}A_n^{c}\right) 
&= \prod^{\infty}_{n=N}\P\left(A_n^{c}\right) \\
&= \prod^{\infty}_{n=N}\left(1-\P\left(A_n\right)\right) \\
&\leq \prod^{\infty}_{n=N}e^{-\P(A_n)}\\
&=e^{-\sum^{\infty}_{n=N}\P(A_n)}\\
&=e^{-\infty}\\
&= 0,
\end{align*}
and we are done.\\
The first $(\Longrightarrow)$ is the Borel-Cantelli lemma (already proven). So we proved both the $(\Longrightarrow)$'s implications. Now it's just a matter of elementary logic to see that the $(\Longleftarrow)$'s hold as well (observe before that the sums must exist since they are sums of non-negative numbers).
\end{proof}

\paragraph{Exercise 7.}
\begin{proof}
\textbf{General remark.} Given two random variables $X,Y$ (which can both assume only finitely-many values) defined on $(\Omega_1,\A_1,\P_1)$ and $(\Omega_2,\A_2,\P_2)$ respectively, the \emph{sum of $X$ and $Y$}, written $X+Y$, is the random variable defined on the product probability space $(\Omega_1 \times \Omega_2, \A_1 \otimes \A_2, \P_1 \otimes \P_2)$ and given by
\[
Z(\omega_1,\omega_2) := X(\omega_1) + Y(\omega_2).
\]
\textbf{Back to the exercise.} Without loss of generality (because what really matters about a random variable is its distribution, not the probability space on which it's defined), we can assume that $X$ is defined on the probability space $(\{1,2\},\PP(\{1,2\}),\P_1)$ and is the identity function, and $Y$ is defined on the probability space $(\{0,1,2\},\PP(\{0,1,2\}),\P_2)$, where $\P_1[\omega] := \frac{1}{2}$ for all $\omega \in \{1,2\}$ and $\P_2$ is defined as in the assignment. Now, the sample space on which $Z$ is defined is by definition
\[
\{1,2\} \times \{0,1,2\}.
\]
By the way the the product $\sigma$-algebra is defined (which I won't discuss here), it is immediate to see that it must contain every singleton. Thus, since the sample space is clearly finite, this means that the $\sigma$-algebra of $Z$ is $\PP(\{1,2\} \times \{0,1,2\})$.\\
By the way the the product probability measure $\P := \P_1 \otimes P_2$ is defined (which I won't discuss here), we must have that
\[
\P[(\omega_1,\omega_2)] = \P_1[\omega_1] \P_2[\omega_2]
\]
for all $\omega_1 \in \{1,2\}$ and $\omega_2 \in \{0,1,2\}$. Since the probability space is clearly finite, this completely determines $\P$ on the whole $\sigma$-algebra (by the additive property).\\
It is left to find the distribution of $Z$. Observe that
\begin{align*}
\P[Z=z] 
&= \P[X+Y=z] \\
&= \P[Y = z -X] \\
&= \P \Big[ [X=1 \text{ and } Y = z-1] \uplus [X=2 \text{ and } Y = z-2] \Big] \\
&= \P[X=1 \text{ and } Y = z-1] + \P[X=2 \text{ and } Y = z-2] \\
&= \P[X=1] \P[Y=z-1] + \P[X=2]\P[Y=z-2] \tag{by independence} \\
&= \frac{1}{2} \P[Y=z-1] + \frac{1}{2} \P[Y = z-2]. \tag{$*$}
\end{align*}
Since $Y$ can take values only among $\{0,1,2\}$, $(*)$ can be different from $0$ only if $z=1,2,3,4$ (which \emph{of course} makes sense, doesn't it?). Now the last computations (which I am not gonna do):
\begin{itemize}
\item $\P[Z=1] = ...$
\item $\P[Z=2] = ...$
\item $\P[Z=3] = ...$
\item $\P[Z=4] = ...$
\end{itemize}
\end{proof}

\paragraph{Exercise 8.}\
\\
$\P[X=0] = 0.03 + 0.16 + 0.12 = 0.31$.\\
$\P[X=1] = ... = 0.69$.\\
$\P[Y=-1] = 0.03 + 0.07 = 0.10$.\\
$\P[Y=2] = ... = 0.51$.\\
$\P[Y=3] = ... = 0.39$.\\
$X$ and $Y$ are not independent, because e.g. $\P[(X,Y)=(0,-1)] = 0.03 \neq 0,031 = 0.31 \cdot 0.10 = \P[X=0] \cdot \P[Y=-1]$.

\paragraph{Exercise 9.}
\begin{proof}
Let $\mu := \E[X]$. Since $X$ is discrete, it can take at most countable-many values $\{x_n \mid n \in \N\}$. Recall that $\Var[X] := \E[(X-\mu)^2] = \sum_{n \in \N} (x_n-\mu)^2 \cdot \P[X=x_n]$. %Without loss of generality, we can suppose that $\P[X=x_n]>0$ for all $n \in \N$ (otherwise we just remove the terms from the sum\footnote{Observe that we can't end up removing \emph{all} terms, because there must be at least one $x_n$ such that $\P[X=x_n]>0$, since $\sum_{n \in \N} \P[X=x_n] = 1$.}). 
Now observe that the terms in the sum are all $\geq 0$. If $\Var[X]=0$, it clearly means that they are all $0$. So
\[
(x_n - \mu)^2 \P[X=x_n] = 0 \text{ for all } n \in \N.
\]
If $\P[X=x_n]=0$, we ``store'' $x_n$ in the set of all the values with $0$ probability. This whole set clearly has probability $0$, and this is where the ``almost surely'' in the statement comes from.\\
If $\P[X=x_n]>0$, it means that $(x_n - \mu)^2=0$, i.e. $x_n = \mu$.\\
So $X(\omega) = \mu$ except for a set of probability $0$, which is exactly what we must prove.
\end{proof}

\paragraph{Exercise 10.} Observe that
\begin{align*}
E \left[ \frac{1}{X+1} \right]
&= \sum_{k=0}^n \frac{1}{k+1} \P[X=k] \\
&= \sum_{k=0}^n \frac{1}{k+1} \binom{n}{k} p^k (1-p)^{n-k} \\
&= \sum_{k=0}^n \frac{1}{k+1} \frac{n!}{k!(n-k)!} p^k (1-p)^{n-k} \\
&= \sum_{k=0}^n \frac{n!}{(k+1)!(n-k)!} p^k (1-p)^{n-k} \\
&= \frac{1}{(n+1)p} \sum_{k=0}^n \frac{(n+1)!}{(k+1)!(n-k)!} p^{k+1} (1-p)^{n-k} \\
&= \frac{1}{(n+1)p} \sum_{k=0}^n \frac{(n+1)!}{(k+1)!(n-k)!} p^{k+1} (1-p)^{n-k} \\
&= \frac{1}{(n+1)p} \sum_{k=0}^n \binom{n+1}{k+1} p^{k+1} (1-p)^{n-k} \\
&= \frac{1}{(n+1)p} \left( \sum_{k=-1}^n \binom{n+1}{k+1} p^{k+1} (1-p)^{n-k} \right) - \frac{1}{(n+1)p} \binom{n+1}{0} p^0 (1-p)^{n+1}\\
&= \frac{1}{(n+1)p} \left( \sum_{k'=0}^{n+1} \binom{n+1}{k'} p^{k'} (1-p)^{n-(k'-1)} \right) - \frac{1}{(n+1)p} (1-p)^{n+1}\\
&= \frac{1}{(n+1)p} \left( \sum_{k'=0}^{n+1} \underbrace{\binom{n+1}{k'} p^{k'} (1-p)^{n+1-k'}}_{=:\phi(k)} \right) - \frac{1}{(n+1)p} (1-p)^{n+1}\\
&= \frac{1}{(n+1)p} - \frac{1}{(n+1)p} (1-p)^{n+1} \tag{$*$} \\
&= \frac{1-(1-p)^{n+1}}{(n+1)p},
\end{align*}
where $(*)$ holds because $\phi(k)$ is the distribution function of $\operatorname{Bin}(n+1,\frac{1}{2})$, and so its sum from $0$ to $n+1$ is $1$.

\paragraph{Exercise 11.} Note: I am pretty sure that we also need the hypothesis $X \geq 0$, i.e. $x_i \geq 0$ for all $i \in \N$, like in the continuous case seen in class. Moreover, we also need the hypothesis $\E[X] \neq 0$, otherwise the statement is clearly false (this is because our statement is slightly different from the standard one, but the professor forgot to add this hypothesis). However:
\begin{proof}
Call $\mu := \E[X]$.
\begin{align*}
\E[X] 
&= \sum_{i \in \N} x_i \P[X=x_i] \\
&= \sum_{x_i < a\mu} x_i \P[X=x_i] + \sum_{x_i \geq a\mu} x_i \P[X=x_i] \\
&\geq \sum_{x_i \geq a\mu} x_i \P[X=x_i] \tag{because $X>0$} \\
&\geq \sum_{x_i \geq a\mu} a\mu \P[X=x_i] \\
&= a\mu \sum_{x_i \geq a\mu} \P[X=x_i] \\
&= a\mu \P[X \geq a\mu].
\end{align*}
\end{proof}























\end{document}
